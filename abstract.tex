The North Western Sahara Aquifer System stands out as one of the water scarcest regions in the world. Moreover, in recent decades agriculture activity has grown exacerbating the pressure on groundwater resources and pumping energy requirements. In this study, a nexus approach was used to assess the effect of capturing, treating and reusing wastewater for irrigation. GIS-based tools were used to capture the systems spatial dimension, enabling to match wastewater supply and water demand points, identify demand hotspots and evaluate techno-economically viable wastewater treatment options. Moreover, the minimum energy requirements for brackish water desalination were estimated. Eight wastewater treatment technologies were evaluated, making use of a Levelized Cost of Water methodology to identify the least-cost system. Six scenarios were constructed based on population water requirements and water-consumption behaviour of farmers towards changes in irrigation water pricing. The identified least-cost technologies showed clear trade-offs related to treatment capacity. The reuse of treated wastewater in agricultural irrigation, showed improvement of groundwater stress, reducing up to 33\% water abstractions and lowering stress levels from medium-to-high to low-to-medium. Furthermore, reuse of wastewater decreased dependency on groundwater pumping and the overall energy-for-water requirements. However, to effectively preserve water resources and act holistically towards the sustainable development agenda, measures as better water pricing mechanisms, management strategies to improve water productivity and adoption of more efficient irrigation schemes may be needed.