The North Western Sahara Aquifer System stands out as one of the water scarcest regions in the world. Moreover, in recent decades agriculture activity has grown exacerbating the pressure on groundwater resources and pumping energy requirements. In this study, a water-energy-food Nexus approach was used to assess the effect of capturing, treating and reusing wastewater for irrigation. GIS-based tools were used to capture the systems spatial dimension, enabling to match wastewater supply and water demand points, identify demand hotspots and evaluate techno-economically viable wastewater treatment options. Moreover, the minimum energy requirements for brackish water desalination were estimated. Seven domestic wastewater treatment technologies and one irrigation tailwater treatment technology were evaluated, making use of a Levelized Cost of Water methodology to identify the least-cost system. Four scenarios were constructed based on water-consumption behaviour of farmers towards changes in irrigation water pricing. The identified least-cost wastewater treatment technologies showed clear trade-offs, as different technologies were more cost-effective depending on treatment capacity requirements of the spatially distributed agglomerations. The reuse of treated wastewater/tailwater in agricultural irrigation, showed improvement of groundwater stress, reducing on about 46\% water abstractions and groundwater stress levels in the best case scenario. However, groundwater stress still fell on the extremely high category, highlighting the critical condition of the aquifer. Furthermore, reuse of wastewater/tailwater decreased dependency on groundwater pumping and the overall energy-for-water requirements, reducing by about 12\% the total energy requirements in the best case scenario. However, to effectively preserve water resources and act holistically towards the sustainable development agenda, measures as better water pricing mechanisms, management strategies to improve water productivity and adoption of more efficient irrigation schemes may be needed.