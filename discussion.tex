On the discussion  explain well how the findings are relevant for the region, but for other regions in the world as well. This can be covered on a broader perspective of the nexus and touch on the circularity of the resources also mentioning the potential for nutrient recovery and energy generation (cite paper from UNIFLORES). Touch some examples as the Egyptian one, on wastewater resource management. The main story-line here is that by capturing treating and reusing wastewater, a more circular use of the resource is being used, accounting for several uses. The potential connection with SDGs could also be mentioned, as SDG for water and sanitation, SDG for energy, SDG for health and SDG for partnership. The paper on Nexus perspectives towards the SDGs can be usd as reference and use the 5 node methodology proposed there to analyse the implications on a systems perspective. The land use and water recovery systems can be analysed, pointing at the interlinkage between land use and water storage. Also analyse the implications on water and food security, both for the region an in a wider context. Then talk about the implications on energy consumption and the connection with climate change due to the high fossil fuel based power sector of the region. Finally finish by acknowledging the potential of treating and reusing wastewater, but the need of having strong water management and pricing mechanisms coupled with technological improvements on agricultural practices as well as the perception of wastewater as a resource, so water stress can be really alleviated in the region and more generally in the world. It is important to discuss as well the importance of cooperation in the region for the proper management of resources (included wastewater).