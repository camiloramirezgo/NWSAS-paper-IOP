Wastewater reclaim, treatment and reuse, has a clear potential to alleviate water stress in the NWSAS, while supporting sustainable food production and energy efficiency. Supported by a GIS-based quantitative analysis, the use of a Nexus approach sheds light on synergies and trade-offs among selected Sustainable Development targets of interest. Wastewater treatment and reuse could indirectly improve water supply (SDG 6.1) as it can increase water availability, given that proper regulations are adopted to prevent increments in sectoral water usage. Water quality (SDG 6.3) is directly enhanced and water efficiency (SDG 6.4) clearly improved as shown by the GWS. Food security (SDG 2.1/2.2) and agricultural production are directly supported by adopting sustainable practices (e.g. as tailwater reclaim and treatment) and reducing soil salinization (i.e. due to reduction of untreated wastewater/tailwater discharged to the environment). Energy efficiency (SDG 7.3) can be positively affected, as wastewater treatment showed to be less energy intense than pumping water from the groundwater aquifer. Thus, by reusing treated wastewater the overall use of energy per cubic meter delivered could be reduced. This, however, is dependent on the wastewater reclaim and treated wastewater supply system. If large centralized systems are implemented, the need to cover long distances from supply to demand may be inevitable (i.e. wastewater collection points to agricultural irrigation sites), thus large amounts of energy may be required for surface water pumping. Nonetheless, by using decentralized systems (e.g. on-farm treatment and wastewater treatment in small agglomerations), those issues can be lessen. 
% Renewable energy share (SDG 7.2) can be positively affected too, as with modern wastewater treatment systems with on-site energy generation (e.g. biogas capture from sludge), large parts of wastewater treatment energy requirements can be provided, increasing the overall share of renewable energy. 
Moreover, by supplying treated wastewater to farmers, on-farm pumping from groundwater aquifers can be reduced, reducing as well the use of fossil fuels and electricity from the grid---the electricity generation mix in the NWSAS countries is heavily dominated by fossil fuel sources, making it an unclean source of energy \cite{AlgeriaElectricityHeat2016,LibyaElectricityHeat2016,TunisiaElectricityHeat2016}. Accordingly, climate mitigation can also be improved as both the water and the agricultural sectors GHG emissions would be reduced. Moreover, climate resilience (SDG 13.1) also has potential to be improved, as the increase on water availability for agricultural production (and indirectly for drinking purposes), would aid on severe drought periods. 
% Synergies with SDG 8 on sustained, inclusive and sustainable economic growth, full and productive employment and decent work for all, and SDG 12 on responsible consumption and production could also be found. This was argued by \citet{hoffNexusApproachMENA2019}, who covered 2 case studies of wastewater reuse in the MENA region and identified similar synergies as the ones previously presented.

Despite the synergies and high potential treated wastewater has in supporting sustainable development and alleviating water scarcity, it can be argued that the measure of increased wastewater reuse alone is not enough. As a matter of facts, none of the scenarios integrating treated wastewater/tailwater reuse achieved a reduction on groundwater stress category. A key parameter affecting such indicator was the water use behaviour of farmers towards price regimes. The inappropriate valorization of water in the NWSAS has been already identified, as well as the inefficiency of irrigation \cite{BetterValorizationIrrigation2015}. Therefore, water management strategies and proper pricing mechanisms that ensure the appropriate use of the resource by local farmers are needed. Moreover, the perception of wastewater reuse from local farmers is key on achieving successful strategies, as in some cases this has shown to be an important barrier for treated wastewater reuse \cite{mahjoubPublicAcceptanceWastewater2018}. As \citet{mahjoubPublicAcceptanceWastewater2018} analyze ``aspects related to education, knowledge, risk perception, culture, regulation, and communication need to be seriously addressed for a more viable and efficient use of wastewater in agriculture".

There is great value on having a Nexus approach while evaluating the treated wastewater reuse measure. It enables to identify and potentiate synergies and mitigate or avoid trade-offs. Nexus thinking has the potential to enhance well-being by decoupling it from natural resources degradation. Special attention needs to be put into understanding the cultural and social characteristics from the evaluated region, and the mechanism and strategies needed to create awareness, acceptance and implementation. Thus, articulated policies, regulations, and monitoring mechanisms are needed to properly valuate and manage natural resources holistically rather than in silos and achieve the full potential that treated wastewater reuse poses. Moreover, the use of an enhanced LCOW methodology including tax factors, externalities and specific discount rates, could aid on understanding the efforts required to make treated wastewater/tailwater competitive against local costs of water (what the farmer actually pays).

Finally, limitations exist in the current study. Although sensitivity analysis on selected input variables was provided in the \textit{supplementary material}, uncertainty in other input data exists. Crop water requirements my be very sensible to variations in climatic variables as temperature, wind speed, solar irradiation, and precipitation. Therefore, proper climatic projections can be used in combination with detailed hydrological models to estimate water availability, compute water requirements for crop irrigation and estimate groundwater level change. Moreover, inputs related to crops harvested, wastewater treatment costs, water use per capita, coverage of wastewater sanitation, wastewater pollutant composition, energy intensity of treatment technologies, recoverable wastewater and agricultural drainage, among others, can be improved with site-specific data. However, as the aim of this study was to explore, from a Nexus perspective, the  sustainable development implications of reusing treated wastewater in irrigation, the authors believe that the present analysis is valuable to promote sustainable and holistic measures to ease water scarcity and could support the execution of more detailed and site-specific developments.