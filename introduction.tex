% Add selected references from 2 fonts: other nexus assessments looking at wastewater, other assessments and literature on NWSAS. The idea here is that you show that there are some clear gaps in literature, that get you to your research questions, that you answer in the paper. As starting point the next text can help for the studies taken in the NWSAS. It will need some edition as some parts were used for the previous paper. Also, should I cite our work on the NWSAS too? moreover, it is necessary to explain the transboundary characteristic of the basin and why is it key for the region.

% Start with introduction on water scarcity issue and literature on wastewater as a resource, covering its current state and reuse potential. Link this literature to the nexus-SDGs nature publications, highlighting the key connections of the water SDG to the health and sanitation SDG (for which wastewater treatment is key) and the direct link to energy and food (due to water availability for irrigation). In here mention some wastewater-energy nexus perspective worldwide. Then, look into the assessments and literature in the region and the wastewater treatment and reuse studies (Mediterranean and NWSAS). Finally, look into the wef nexus studies of the region (MENA and NWSAS) and explain gaps with nexus on wastewater-energy-food perspective both globally and in the region, and highlight the main objective of the research.

Globally, variations on the dynamics of the water cycle are causing disparity in water demand and supply \cite{FAO2015,unescoWastewaterUntappedResource2017}. This has led to recurrent and prolonged drought periods pushing two-thirds of the worlds population to experience water scarcity for at least one month a year \cite{Mekonnene1500323}. This problematic makes utterly challenging to cope with an expected increase in water demand due to agricultural, industrial and population growth over the next decades \cite{IFPRI2017}. Moreover, inefficient use of water aggravates the problem and puts more pressure into the system. Worldwide, agriculture irrigation consumes around 70\% of all bluewater resources, of which 45\% are not used and returned as drainage \cite{unescoWastewaterUntappedResource2017}. This share is even higher for population wastewater, as around 70\% of bluewater abstractions for population are returned as wastewater \cite{unescoWastewaterUntappedResource2017}. Overall, 56\% of all bluewater abstractions are wasted, however, this resource still carries a vast potential. Nutrients can be recovered from wastewater and subsequently used as high-quality fertilizers for food production [REF]. Energy can be extracted in the form of biogas through bioreaction processes and used for heat and electricity production [REF]. Whereas by applying proper wastewater treatment processes, the resulting water stream can be used for agricultural irrigation, ecosystems services maintenance or even human consumption [REF].

Wastewater treatment and reuse has been internationally recognized as one of the best measures to ease water scarcity as it can substantially increase water availability \cite{unescoWastewaterUntappedResource2017,GARCIA2015154}. This---integrated with water resources management and expanded efforts in water harvesting, desalination and water efficiency---strongly supports Sustainable Development Goal number 6, specifically contributing to targets 6.5 and 6.A \cite{UNSDGs2019}. According to the AQUASTAT database, the amount of industrial and municipal wastewater treated worldwide, ranges from 8\% to 70\% in low-income and high-income countries respectively. However, agricultural drainage water is almost never collected or treated, and only a marginal amount of the treated wastewater is actually reused \cite{unescoWastewaterUntappedResource2017}. Non-treated wastewater that runs freely to the environment, often poses severe environmental and health consequences, polluting groundwater aquifers, rivers, lakes, soils and food among others. Nonetheless, there are cases of success where treated wastewater reuse has supported food production, ecosystem services and aquifer recharge.

In Egypt....
In Jordan...
In USA...
In north Tunisia...
In Europe...

Scientific characterization of the NWSAS basin has been taking place since the 1960s, gaining important momentum in the 1980s (AbuZeid and Elrawady, 2015). The main studies have been conducted in the form of joint efforts between Algeria and Tunisia, and Libya joining later. Such efforts have identified challenges and risks that the NWSAS has been facing mainly in terms of water scarcity and utilization. In 1999 a multi-phase project led by the Sahara and Sahel Observatory (OSS), launched with the purpose of ensuring control over potential impacts of transboundary water resources (Khater, 2014). The first phase of the program ended in year 2002, achieving a common database containing over 9,000 water points, a hydraulic model to assess impacts of water withdrawals, and a consultation mechanism for joint management of water resources that evolved to a permanent structure in 2008 (AbuZeid and Elrawady, 2015; Khater, 2014). The modelling efforts undertaken in the first phase of the project were aimed at identifying new water withdrawal potential in the aquifer and the associated risk of future exploitation, projecting scenarios to year 2050 (Khater, 2014). Moreover, from year 2003, new studies were carried out as part of the second phase of the project, in order to make a diagnosis of the agriculture sector, in reference to the hydraulic aspects of the basin (OSS, 2015). The study helped to identify the inefficiency of irrigation, the inadequate valorization of water and the degradation of soil quality in the NWSAS.

In 2009, OSS continued with the third phase of the project developing a study on water valorization in the NWSAS, focused on enhancing sustainable development in the region (OSS, 2015). The study consisted of two main components: a socio-economic component that analyzes the modes of operation of agricultural systems and understanding the behaviour of the irrigators (OSS, 2014b). For this, several surveys were conducted to almost 3,000 farmers. The second component is known as “demonstration pilots”, consisted on implementing technical innovations aimed at testing their feasibility and acceptance on the farmers level, with the purpose of saving and valorizing water in the NWSAS basin (OSS, 2014c). These pilots were comprised of six projects that focused on certain areas (oases) and can be catalogued into four themes: 1) solar energy for pumping water, both for irrigation and for drainage evacuation, 2) brackish water valuation through demineralization, 3) rehabilitation of lands degraded due to water stagnation, and 3) irrigation efficiency and agricultural intensification.

Although the joint efforts of the three countries and the OSS have been substantial and key for understanding the complexity and vulnerability of the NWSAS system, the components of the research have been heavily driven by understanding water issues---with a lesser focus on the sectors that drive or are needed to support the supply of that water. Another limitation is that most of the pilot studies covered certain areas on the NWSAS and no study really considered the entire NWSAS especially when it comes to energy and water aspects. In this regard, an evaluation of the resources in a nexus context has not yet been performed. A regional study that tackles interlinkages between the agriculture, water and energy sectors in the NWSAS, is necessary to achieve a joint planning and sustainable development of the three sectors in the basin and all the implicated countries. 
\cite{gremillionWastewaterResourceWaterWasteEnergy}