Add selected references from 2 fonts: other nexus assesments looking at wastewater, other assesments and literature on NWSAS. The idea here is that you show that there are some clear gaps in literature, that get you to your research questions, that you answer in the paper. As starting point the next text can help for the studies taken in the NWSAS. It will need some edition as some parts were used for the previous paper. Also, should I cite our work on the NWSAS too?

Scientific characterization of the NWSAS basin has been taking place since the 1960s, gaining important momentum in the 1980s (AbuZeid and Elrawady, 2015). The main studies have been conducted in the form of joint efforts between Algeria and Tunisia, and Libya joining later. Such efforts have identified challenges and risks that the NWSAS has been facing mainly in terms of water scarcity and utilization. In 1999 a multi-phase project led by the Sahara and Sahel Observatory (OSS), launched with the purpose of ensuring control over potential impacts of transboundary water resources (Khater, 2014). The first phase of the program ended in year 2002, achieving a common database containing over 9,000 water points, a hydraulic model to assess impacts of water withdrawals, and a consultation mechanism for joint management of water resources that evolved to a permanent structure in 2008 (AbuZeid and Elrawady, 2015; Khater, 2014). The modelling efforts undertaken in the first phase of the project were aimed at identifying new water withdrawal potential in the aquifer and the associated risk of future exploitation, projecting scenarios to year 2050 (Khater, 2014). Moreover, from year 2003, new studies were carried out as part of the second phase of the project, in order to make a diagnosis of the agriculture sector, in reference to the hydraulic aspects of the basin (OSS, 2015). The study helped to identify the inefficiency of irrigation, the inadequate valorization of water and the degradation of soil quality in the NWSAS.

In 2009, OSS continued with the third phase of the project developing a study on water valorization in the NWSAS, focused on enhancing sustainable development in the region (OSS, 2015). The study consisted of two main components: a socio-economic component that analyzes the modes of operation of agricultural systems and understanding the behaviour of the irrigators (OSS, 2014b). For this, several surveys were conducted to almost 3,000 farmers. The second component is known as “demonstration pilots”, consisted on implementing technical innovations aimed at testing their feasibility and acceptance on the farmers level, with the purpose of saving and valorizing water in the NWSAS basin (OSS, 2014c). These pilots were comprised of six projects that focused on certain areas (oases) and can be catalogued into four themes: 1) solar energy for pumping water, both for irrigation and for drainage evacuation, 2) brackish water valuation through demineralization, 3) rehabilitation of lands degraded due to water stagnation, and 3) irrigation efficiency and agricultural intensification.

Although the joint efforts of the three countries and the OSS have been substantial and key for understanding the complexity and vulnerability of the NWSAS system, the components of the research have been heavily driven by understanding water issues---with a lesser focus on the sectors that drive or are needed to support the supply of that water. Another limitation is that most of the pilot studies covered certain areas on the NWSAS and no study really considered the entire NWSAS especially when it comes to energy and water aspects. In this regard, an evaluation of the resources in a nexus context has not yet been performed. A regional study that tackles interlinkages between the agriculture, water and energy sectors in the NWSAS, is necessary to achieve a joint planning and sustainable development of the three sectors in the basin and all the implicated countries. 
\cite{gremillionWastewaterResourceWaterWasteEnergy}