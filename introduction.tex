% Add selected references from 2 fonts: other nexus assessments looking at wastewater, other assessments and literature on NWSAS. The idea here is that you show that there are some clear gaps in literature, that get you to your research questions, that you answer in the paper. As starting point the next text can help for the studies taken in the NWSAS. It will need some edition as some parts were used for the previous paper. Also, should I cite our work on the NWSAS too? moreover, it is necessary to explain the transboundary characteristic of the basin and why is it key for the region.

% Start with introduction on water scarcity issue and literature on wastewater as a resource, covering its current state and reuse potential. Link this literature to the nexus-SDGs nature publications, highlighting the key connections of the water SDG to the health and sanitation SDG (for which wastewater treatment is key) and the direct link to energy and food (due to water availability for irrigation). In here mention some wastewater-energy nexus perspective worldwide. Then, look into the assessments and literature in the region and the wastewater treatment and reuse studies (Mediterranean and NWSAS). Finally, look into the wef nexus studies of the region (MENA and NWSAS) and explain gaps with nexus on wastewater-energy-food perspective both globally and in the region, and highlight the main objective of the research.

It is well known that variations on the dynamics of the water cycle are causing disparity in water demand and supply globally \cite{FAO2015,unescoWastewaterUntappedResource2017}. This, has led to recurrent and prolonged drought periods pushing two-thirds of the worlds population to experience water scarcity for at least one month a year \cite{Mekonnene1500323}. Such problematic makes utterly challenging to cope with an expected increase in water demand due to agricultural, industrial and population growth over the next decades \cite{IFPRI2017}. Moreover, inefficient use of water aggravates the problem and puts more pressure into the system. Worldwide, agriculture irrigation consumes around 70\% of all bluewater resources, of which 45\% are not used and returned as untreated drainage \cite{unescoWastewaterUntappedResource2017}. In addition, around 70\% of bluewater abstractions for population uses are returned as wastewater \cite{unescoWastewaterUntappedResource2017}. Overall, 56\% of all bluewater abstractions are wasted, however, this resource still carries a vast potential. Nutrients can be recovered from wastewater and subsequently used as high-quality fertilizers for food production \cite{moEnergyNutrientsWater2013a}. Heat and electricity can be generated by using extracted biogas from anaerobic digestion processes, biosolids incineration, effluents hydropower, heat pumps and bioelectrochemical systems, among others \cite{moEnergyNutrientsWater2013a}. Whereas, high quality treated wastewater can be reused for agricultural irrigation, ecosystems services maintenance, industrial processes and groundwater replenishment \cite{moEnergyNutrientsWater2013a}.

Wastewater treatment and reuse has been internationally recognized as one of the best measures to ease water scarcity as it can substantially increase water availability \cite{unescoWastewaterUntappedResource2017,GARCIA2015154}. According to the AQUASTAT database, the amount of industrial and municipal wastewater treated worldwide, ranges from 8\% to 70\% in low-income and high-income countries respectively. However, agricultural drainage water is almost never collected or treated, and only a marginal amount of the treated wastewater is actually reused \cite{unescoWastewaterUntappedResource2017}. Non-treated wastewater that runs freely to the environment, often poses severe environmental and health consequences, polluting groundwater aquifers, rivers, lakes, soils and food, among others. Nonetheless, there are several cases of success where treated wastewater reuse has substantially improved agricultural irrigation (aiding food production), incremented water availability for ecosystem services and supported aquifer recharge \cite{hettiarachchiSAFEUSEWASTEWATERa, halalshehPolicyGovernanceFramework2018, mahjoubPublicAcceptanceWastewater2018, zuurbierUseWastewaterManaged2018, hussainSustainableUseManagement2019}.

Wastewater reuse plays an important role for sustainable development as it is essential for achieving Sustainable Development Goal (SDG) 6 on clean water and sanitation. Key targets of SDG 6 are to improve water quality by reducing by half the amount of non-treated wastewater and increasing recycling and safe reuse globally \cite{UNSDGs2019, tortajadaContributionsRecycledWastewater2020}. The achievement of such targets can substantially increase availability of clean water resources for all uses, not available otherwise. This also means that discharge of cleaner wastewater in ecosystems can support SDG 14 on life below water and SDG 15 on life on land \cite{UNSDGs2019, tortajadaContributionsRecycledWastewater2020}. Furthermore, synergies and trade-offs between the achievement of SDG6 and with virtually all SDGs exist \cite{WaterSanitationInterlinkages2016}. The strong interconnection between SDGs and their underlining systems \cite{WaterSanitationInterlinkages2016,fusoneriniMappingSynergiesTradeoffs2018,fusoneriniConnectingClimateAction2019}, supports the use of holistic approaches to evaluate actions towards achieving sustainable development targets, commonly known as Nexus approaches \cite{liuNexusApproachesGlobal2018,bleischwitzResourceNexusPerspectives2018,olawuyiSustainableDevelopmentWaterenergyfood2020,simpsonDevelopmentWaterEnergyFoodNexus2019,hoffNexusApproachMENA2019}. Links between wastewater reuse and SDG 2 on zero hunger (mainly due to sustainable food production and resilient agricultural practices), SDG 7 on affordable and clean energy (mainly due to improved energy efficiency and increasing the share of renewable energy) and SDG 13 on climate action (mainly due to strengthen resilience and adaptive capacity to climate-related hazards), can be noticed as a clear water, energy, food and climate Nexus \cite{WaterSanitationInterlinkages2016,liuNexusApproachesGlobal2018,hoffNexusApproachMENA2019}.

Nexus approaches have been widely used for evaluating interlinkages between resource systems, trying to identify challenges, synergies, trade-offs and evaluate holistic solutions, specially in the MENA region \cite{hoffNexusApproachMENA2019,kingRapidAssessmentWater2015}. Furthermore, Nexus thinking is also gaining attention in transboundary resources settings \cite{uneceReconcilingResourceUses2015,kingRapidAssessmentWater2015}. Such is the case of the North Western Sahara Aquifer System (NWSAS). The NWSAS is located in North Africa covering large parts of Algeria, Tunisia and Libya, and it holds invaluable groundwater resources to maintain livelihood in the region \cite{BetterValorizationIrrigation2015}.

The main scientific studies on the NWSAS have been conducted in the form of joint efforts between Algeria, Tunisia, and Libya \cite{abuzeidNorthWesternSahara2015}. Such efforts have identified challenges and risks that the NWSAS has been facing mainly in terms of water scarcity and utilization \cite{BetterValorizationIrrigation2015}. Research outcomes have achieved common databases containing over 9,000 water points, developed hydraulic models to assess impacts of water withdrawals, set consultation mechanisms for joint management of water resources, evaluated new water withdrawal potential and the associated risk of future exploitation, identified the inefficiency of irrigation, the inadequate valorization of water and the degradation of soil quality in the region \cite{khaterNorthWesternSahara2014,abuzeidNorthWesternSahara2015,BetterValorizationIrrigation2015,Socioeconomicaspectsirrigation2014}. Moreover, technical innovation pilots have been implemented in the NWSAS under four themes \cite{ossAgriculturalDemostrationPilots2014}: 1) solar energy for pumping water, both for irrigation and for drainage evacuation, 2) brackish water valuation through demineralization, 3) rehabilitation of lands degraded due to water stagnation, and 3) irrigation efficiency and agricultural intensification. 

Although, the joint efforts of the three countries and the OSS have been substantial and key for understanding the complexity and vulnerability of the NWSAS system, the components of the research have been heavily driven by understanding water issues---with a lesser focus on the sectors that drive or are needed to support the supply of water. In addition, the wastewater reclaim treatment and reuse potential has not yet been explored at the basin level, neither the synergies and trade-offs it may have with the energy, water and agricultural sectors. In general, wastewater treatment and reuse measures have been commonly evaluated with a water centred approach, often overlooking energy requirements and its implications. On the other hand, water-energy nexus approaches exploring the wastewater topic, have usually focused on the energy and nutrients recovery potential from wastewater. Moreover, feasibility studies of reusing wastewater, have assessed potential on specific sites (e.g. wastewater treatment plants) or at the national level based on general statistics. To date, a research gap was found into how to determine the potential for wastewater treatment, reclaim and reuse, taking into account potential treatment technologies, energy demand implications and effects on the water system. To address that gap, in this paper a Nexus approach is used to asses the impact that reclaiming, treating and reusing wastewater (i.e. in agricultural irrigation), may have in the water, food and energy systems in the NWSAS, making use of Geographic Information Systems (GIS) and a Levelised Cost of Water (LCOW) methodology.

% In 1999 a multi-phase project led by the Sahara and Sahel Observatory (OSS), launched with the purpose of ensuring control over potential impacts of transboundary water resources (Khater, 2014). The first phase of the program ended in year 2002, achieving a common database containing over 9,000 water points, a hydraulic model to assess impacts of water withdrawals, and a consultation mechanism for joint management of water resources that evolved to a permanent structure in 2008 (AbuZeid and Elrawady, 2015; Khater, 2014). The modelling efforts undertaken in the first phase of the project were aimed at identifying new water withdrawal potential in the aquifer and the associated risk of future exploitation, projecting scenarios to year 2050 (Khater, 2014). Moreover, from year 2003, new studies were carried out as part of the second phase of the project, in order to make a diagnosis of the agriculture sector, in reference to the hydraulic aspects of the basin (OSS, 2015). The study helped to identify the inefficiency of irrigation, the inadequate valorization of water and the degradation of soil quality in the NWSAS.

% In 2009, OSS continued with the third phase of the project developing a study on water valorization in the NWSAS, focused on enhancing sustainable development in the region (OSS, 2015). The study consisted of two main components: a socio-economic component that analyzes the modes of operation of agricultural systems and understanding the behaviour of the irrigators (OSS, 2014b). For this, several surveys were conducted to almost 3,000 farmers. The second component is known as “demonstration pilots”, consisted on implementing technical innovations aimed at testing their feasibility and acceptance on the farmers level, with the purpose of saving and valorizing water in the NWSAS basin (OSS, 2014c). These pilots were comprised of six projects that focused on certain areas (oases) and can be catalogued into four themes: 1) solar energy for pumping water, both for irrigation and for drainage evacuation, 2) brackish water valuation through demineralization, 3) rehabilitation of lands degraded due to water stagnation, and 3) irrigation efficiency and agricultural intensification.

% Although the joint efforts of the three countries and the OSS have been substantial and key for understanding the complexity and vulnerability of the NWSAS system, the components of the research have been heavily driven by understanding water issues---with a lesser focus on the sectors that drive or are needed to support the supply of that water. Another limitation is that most of the pilot studies covered certain areas on the NWSAS and no study really considered the entire NWSAS especially when it comes to energy and water aspects. In this regard, an evaluation of the resources in a nexus context has not yet been performed. A regional study that tackles interlinkages between the agriculture, water and energy sectors in the NWSAS, is necessary to achieve a joint planning and sustainable development of the three sectors in the basin and all the implicated countries. 
% \cite{gremillionWastewaterResourceWaterWasteEnergy}