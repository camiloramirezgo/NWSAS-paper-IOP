The uncontrolled growth of demand for natural resources seen over the last decades, has made deeply challenging to keep natural resources exploitation under planetary boundaries \cite{planetaryBoundaries2009}. This, has entailed severe repercussions over human livelihoods, natural ecosystems and life on earth \cite{planetaryBoundaries2009,Steffen1259855}. To control such behaviour and address sustainable societal growth, in year 2015 the member states of United Nations (UN) adopted the 17 Sustainable Development Goals (SDGs) \cite{rosaTransformingOurWorld2017}. To accomplish such goals, it has been proved essential to account for the strong interconnection between SDGs and their underlining systems \cite{WaterSanitationInterlinkages2016,fusoneriniMappingSynergiesTradeoffs2018,fusoneriniConnectingClimateAction2019}. 

SDG 6 on clean water and sanitation has been one of the center points of sustainable development. Variations on the dynamics of the water cycle are causing disparity in water demand and supply globally \cite{FAO2015,unescoWastewaterUntappedResource2017}. This, has led to recurrent and prolonged drought periods pushing two-thirds of the worlds population to experience water scarcity for at least one month a year \cite{Mekonnene1500323}. Such problematic makes utterly challenging to cope with an expected increase in water demand due to agricultural, industrial and population growth over the next decades \cite{IFPRI2017}. Moreover, inefficient use of water aggravates the problem and puts more pressure into the system.

Wastewater reuse plays an important role for sustainable development as it is essential for achieving SDG 6. Key targets of SDG 6 are to improve water quality by reducing by half the amount of non-treated wastewater and increasing recycling and safe reuse globally \cite{UNSDGs2019, tortajadaContributionsRecycledWastewater2020}. The achievement of such targets can substantially increase availability of clean water resources for all uses not available otherwise, which is why wastewater reuse has been internationally recognized as one of the best measures to ease water scarcity \cite{unescoWastewaterUntappedResource2017,GARCIA2015154}. This also means that discharge of cleaner wastewater in ecosystems can support SDG 14 on life below water and SDG 15 on life on land \cite{UNSDGs2019, tortajadaContributionsRecycledWastewater2020}. 
% According to the AQUASTAT database, the amount of industrial and municipal wastewater treated worldwide, ranges from 8\% to 70\% in low-income and high-income countries respectively. However, agricultural drainage water is almost never collected or treated, and only a marginal amount of the treated wastewater is actually reused \cite{unescoWastewaterUntappedResource2017}. 
Non-treated wastewater that runs freely to the environment, often poses severe environmental and health consequences, polluting groundwater aquifers, rivers, lakes, soils and food, among others \cite{unescoWastewaterUntappedResource2017}. Nonetheless, there are several cases of success where treated wastewater reuse has substantially improved agricultural irrigation (aiding food production), incremented water availability for ecosystem services and supported aquifer recharge \cite{hettiarachchiSAFEUSEWASTEWATERa, halalshehPolicyGovernanceFramework2018, mahjoubPublicAcceptanceWastewater2018, zuurbierUseWastewaterManaged2018, hussainSustainableUseManagement2019}.

Furthermore, synergies and trade-offs between the achievement of SDG 6 and with virtually all SDGs exist \cite{WaterSanitationInterlinkages2016,unescoWastewaterUntappedResource2017}. The inextricable interconnection between SDGs and natural resources, supports the use of holistic approaches to evaluate actions towards achieving sustainable development targets, commonly known as Nexus approaches \cite{liuNexusApproachesGlobal2018,bleischwitzResourceNexusPerspectives2018,olawuyiSustainableDevelopmentWaterenergyfood2020,simpsonDevelopmentWaterEnergyFoodNexus2019,hoffNexusApproachMENA2019}. Links between wastewater reuse and SDG 2 on zero hunger (mainly due to sustainable food production and resilient agricultural practices), SDG 7 on affordable and clean energy (mainly due to improved energy efficiency and increasing the share of renewable energy) and SDG 13 on climate action (mainly due to strengthen resilience and adaptive capacity to climate-related hazards), can be noticed as a clear water, energy, food and climate Nexus \cite{WaterSanitationInterlinkages2016,liuNexusApproachesGlobal2018,hoffNexusApproachMENA2019}.

Nexus approaches have been widely used for evaluating interlinkages between resource systems, trying to identify challenges, synergies, trade-offs and evaluate holistic solutions, specially in the MENA region \cite{hoffNexusApproachMENA2019,kingRapidAssessmentWater2015}. Furthermore, Nexus thinking is also gaining attention in transboundary resources settings \cite{uneceReconcilingResourceUses2015,kingRapidAssessmentWater2015}. Such is the case of the North Western Sahara Aquifer System (NWSAS). The NWSAS is located in North Africa covering large parts of Algeria, Tunisia and Libya, and it holds invaluable groundwater resources to maintain livelihood in the region \cite{BetterValorizationIrrigation2015}.

The main scientific studies on the NWSAS have been conducted in the form of joint efforts between Algeria, Tunisia, and Libya \cite{abuzeidNorthWesternSahara2015}. Such efforts have identified challenges and risks that the NWSAS has been facing mainly in terms of water scarcity and utilization \cite{BetterValorizationIrrigation2015}. Research outcomes have achieved common databases containing over 9,000 water points, developed hydraulic models to assess impacts of water withdrawals, consultation mechanisms for joint management of water resources, 
% evaluated new water withdrawal potential and the associated risk of future exploitation, 
identified the inefficiency of irrigation, the inadequate valorization of water and the degradation of soil quality in the region \cite{khaterNorthWesternSahara2014,abuzeidNorthWesternSahara2015,BetterValorizationIrrigation2015,Socioeconomicaspectsirrigation2014}. Moreover, technical innovation pilots have been implemented in the NWSAS under four themes \cite{ossAgriculturalDemostrationPilots2014}: 1) solar energy for pumping water, both for irrigation and for drainage evacuation, 2) brackish water valuation through demineralization, 3) rehabilitation of lands degraded due to water stagnation, and 4) irrigation efficiency and agricultural intensification. 

Although, the joint efforts of the three countries and the Sahara and Sahel Observatory (OSS) have been substantial and key for understanding the complexity and vulnerability of the NWSAS system, the components of their research have been heavily driven by understanding water issues---with a lesser focus on the sectors that drive or are needed to support the supply of water. \citet{almullaNWSAS} tried to solve that by making the first basin-wide Nexus study, capturing dynamics between the agricultural, water and energy sectors. The focus was on understanding water requirements for agricultural irrigation, its related energy for pumping demand and exploring solar PV competence to provide such energy under various scenarios. The main outcomes of the study are helpful to inform policy making in the three countries and enhance synergies between sectors. However, the aim of the study was not on identifying measures to ease water scarcity, rather than to quantify the Nexus perspective and evaluate options to transition to clean energy sources in agriculture.

The potential of reusing treated wastewater has not yet been explored in the basin, neither the synergies and trade-offs it may have with the energy, water and agricultural sectors. In general, wastewater treatment and reuse has been commonly evaluated with water, food and health centred approaches, often overlooking energy requirements and its implications \cite{unescoWastewaterUntappedResource2017,hettiarachchiSAFEUSEWASTEWATERa,cheniniEvaluationThreeDecades2011,qadirNonconventionalWaterResources2007}. On the other hand, wastewater-related nexus approaches, have usually focused on the energy and nutrients recovery potential from wastewater \cite{guestNewPlanningDesign2009,gremillionWastewaterResourceWaterWasteEnergy,unescoWastewaterUntappedResource2017}. To date, a research gap was found into how to determine the potential of wastewater treatment and reuse, assessing possible treatment technologies, the energy demand implications and the effects on the water system. To address that gap, the objective of this study was to develop a novel exploratory Nexus approach to asses the impact that reclaiming, treating and reusing wastewater (i.e. in agricultural irrigation), may have in the water, food and energy systems. Tailored Geographic Information Systems (GIS) methods were used with a proposed Levelized Cost of Water (LCOW) methodology to capture key spacial characteristics of the Nexus system. Furthermore, the methodology was applied to the NWSAS in the context of sustainable development.