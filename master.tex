%%%%%%%%%%%%%%%%%%%%%%%%%%%%%%%%%%%%%%%%%%%%%%%%%%%%%%%%%%%%%%%%%%%%%%%%
%    INSTITUTE OF PHYSICS PUBLISHING                                   %
%                                                                      %
%   `Preparing an article for publication in an Institute of Physics   %
%    Publishing journal using LaTeX'                                   %
%                                                                      %
%    LaTeX source code `ioplau2e.tex' used to generate `author         %
%    guidelines', the documentation explaining and demonstrating use   %
%    of the Institute of Physics Publishing LaTeX preprint files       %
%    `iopart.cls, iopart12.clo and iopart10.clo'.                      %
%                                                                      %
%    `ioplau2e.tex' itself uses LaTeX with `iopart.cls'                %
%                                                                      %
%%%%%%%%%%%%%%%%%%%%%%%%%%%%%%%%%%
%
%
% First we have a character check
%
% ! exclamation mark    " double quote  
% # hash                ` opening quote (grave)
% & ampersand           ' closing quote (acute)
% $ dollar              % percent       
% ( open parenthesis    ) close paren.  
% - hyphen              = equals sign
% | vertical bar        ~ tilde         
% @ at sign             _ underscore
% { open curly brace    } close curly   
% [ open square         ] close square bracket
% + plus sign           ; semi-colon    
% * asterisk            : colon
% < open angle bracket  > close angle   
% , comma               . full stop
% ? question mark       / forward slash 
% \ backslash           ^ circumflex
%
% ABCDEFGHIJKLMNOPQRSTUVWXYZ 
% abcdefghijklmnopqrstuvwxyz 
% 1234567890
%
%%%%%%%%%%%%%%%%%%%%%%%%%%%%%%%%%%%%%%%%%%%%%%%%%%%%%%%%%%%%%%%%%%%
%
\documentclass[12pt]{iopart}
\newcommand{\gguide}{{\it Preparing graphics for IOP Publishing journals}}
%Uncomment next line if AMS fonts required
%\usepackage{iopams}  
\usepackage{hyperref,tikz,array}
\usepackage[export]{adjustbox}
\usepackage{gensymb,eurosym}
\usepackage[numbers]{natbib}
\usepackage[inline]{enumitem}
\usepackage[all]{hypcap}
\pdfminorversion=4
\graphicspath{{Images/}}
\newcolumntype{P}{>{\raggedright\arraybackslash}p}
\usetikzlibrary{arrows.meta, shapes.geometric}
\newcommand{\rpm}{\raisebox{.3ex}{$\scriptstyle\pm$}}
\hypersetup{hidelinks}

\begin{document}
\title[Reusing wastewater in agriculture: a Nexus assessment in the NWSAS]{Reusing wastewater for agricultural irrigation: a Nexus assessment in the North Western Sahara Aquifer System}

\author{Camilo Ramirez $^{1}$, Youssef Almulla $^{1}$, and Francesco Fuso-Nerini $^{1}$}

\address{$^{1}$ KTH Royal Institute of Technology, Stockholm, Sweden}
\ead{camilorg@kth.se}
\vspace{10pt}
\begin{indented}
\item[]\today
\end{indented}

\begin{abstract}
The North Western Sahara Aquifer System stands out as one of the water scarcest regions in the world. Moreover, in recent decades agriculture activity has grown exacerbating the pressure on groundwater resources and pumping energy requirements. In this study, a water-energy-food Nexus approach was used to assess the effect of capturing, treating and reusing wastewater for irrigation. GIS-based tools were used to capture the systems spatial dimension, enabling to match wastewater supply and water demand points, identify demand hotspots and evaluate techno-economically viable wastewater treatment options. Moreover, the minimum energy requirements for brackish water desalination were estimated. Seven domestic wastewater treatment technologies and one irrigation tailwater treatment technology were evaluated, making use of a Levelized Cost of Water methodology to identify the least-cost system. Four scenarios were constructed based on water-consumption behaviour of farmers towards changes in irrigation water pricing. The identified least-cost wastewater treatment technologies showed clear trade-offs, as different technologies were more cost-effective depending on treatment capacity requirements of the spatially distributed agglomerations. The reuse of treated wastewater/tailwater in agricultural irrigation, showed improvement of groundwater stress, reducing on about 46\% water abstractions and groundwater stress levels in the best case scenario. However, groundwater stress still fell on the extremely high category, highlighting the critical condition of the aquifer. Furthermore, reuse of wastewater/tailwater decreased dependency on groundwater pumping and the overall energy-for-water requirements, reducing by about 12\% the total energy requirements in the best case scenario. However, to effectively preserve water resources and act holistically towards the sustainable development agenda, measures as better water pricing mechanisms, management strategies to improve water productivity and adoption of more efficient irrigation schemes may be needed.
\end{abstract}

% Uncomment for keywords
\vspace{2pc}
\noindent{\it Keywords}: Water-Energy-Food Nexus, wastewater as a resource, agricultural sustainability, NWSAS, GIS-based modelling.

% Uncomment for Submitted to journal title message
% \submitto{\ERL}
%
% Uncomment if a separate title page is required
%\maketitle
% 
% For two-column output uncomment the next line and choose [10pt] rather than [12pt] in the \documentclass declaration
% \ioptwocol
%
\section{Introduction}
% The uncontrolled growth of demand for natural resources seen over the last decades, has made deeply challenging to keep natural resources exploitation under planetary boundaries \cite{planetaryBoundaries2009}. This, has entailed severe repercussions over human livelihoods, natural ecosystems and life on earth \cite{planetaryBoundaries2009,Steffen1259855}. To control such behaviour and address sustainable societal growth, in year 2015 the member states of United Nations (UN) adopted the 17 Sustainable Development Goals (SDGs) \cite{rosaTransformingOurWorld2017}. To accomplish such goals, it has been proved essential to account for the strong interconnection between SDGs and their underlining systems \cite{WaterSanitationInterlinkages2016,fusoneriniMappingSynergiesTradeoffs2018,fusoneriniConnectingClimateAction2019}. 

% SDG 6 on clean water and sanitation has been one of the center points of sustainable development. Variations on the dynamics of the water cycle are causing disparity in water demand and supply globally \cite{FAO2015,unescoWastewaterUntappedResource2017}. This, has led to recurrent and prolonged drought periods pushing two-thirds of the worlds population to experience water scarcity for at least one month a year \cite{Mekonnene1500323}. Such problematic makes utterly challenging to cope with an expected increase in water demand due to agricultural, industrial and population growth over the next decades \cite{IFPRI2017}. Moreover, inefficient use of water aggravates the problem and puts more pressure into the system.

% Wastewater reuse plays an important role for sustainable development as it is essential for achieving SDG 6 on ensuring availability and sustainable management of water and sanitation for all. Key targets of SDG 6 are to improve water quality by reducing by half the amount of non-treated wastewater and increasing recycling and safe reuse globally \cite{UNSDGs2019, tortajadaContributionsRecycledWastewater2020}. The achievement of such targets can substantially increase availability of clean water resources for all uses not available otherwise, which is why wastewater reuse has been internationally recognized as one of the best measures to ease water scarcity \cite{unescoWastewaterUntappedResource2017,GARCIA2015154}. 
% This also means that discharge of treated wastewater in ecosystems can support SDG 14 on life below water and SDG 15 on life on land \cite{UNSDGs2019, tortajadaContributionsRecycledWastewater2020}. 
% According to the AQUASTAT database, the amount of industrial and municipal wastewater treated worldwide, ranges from 8\% to 70\% in low-income and high-income countries respectively. However, agricultural drainage water is almost never collected or treated, and only a marginal amount of the treated wastewater is actually reused \cite{unescoWastewaterUntappedResource2017}. 
% Non-treated wastewater that runs freely to the environment, often poses severe environmental and health consequences, polluting groundwater aquifers, rivers, lakes, soils and food, among others \cite{unescoWastewaterUntappedResource2017}. Nonetheless, there are several cases of success where treated wastewater reuse has substantially improved agricultural irrigation (aiding food production), incremented water availability for ecosystem services and supported aquifer recharge \cite{hettiarachchiSAFEUSEWASTEWATERa, halalshehPolicyGovernanceFramework2018, mahjoubPublicAcceptanceWastewater2018, zuurbierUseWastewaterManaged2018, hussainSustainableUseManagement2019}.

% Furthermore, synergies and trade-offs between the achievement of SDG 6 and with virtually all SDGs exist \cite{WaterSanitationInterlinkages2016,unescoWastewaterUntappedResource2017}. The inextricable interconnection between SDGs and natural resources, supports the use of holistic approaches to evaluate actions towards achieving sustainable development targets, commonly known as Nexus approaches \cite{liuNexusApproachesGlobal2018,bleischwitzResourceNexusPerspectives2018,olawuyiSustainableDevelopmentWaterenergyfood2020,simpsonDevelopmentWaterEnergyFoodNexus2019,hoffNexusApproachMENA2019}. Links between wastewater reuse and SDG 2 on zero hunger (mainly due to sustainable food production and resilient agricultural practices), SDG 7 on affordable and clean energy (mainly due to improved energy efficiency and increasing the share of renewable energy) and SDG 13 on climate action (mainly due to strengthen resilience and adaptive capacity to climate-related hazards), can be noticed as a clear water, energy, food and climate Nexus \cite{WaterSanitationInterlinkages2016,liuNexusApproachesGlobal2018,hoffNexusApproachMENA2019}.

Nexus approaches have been widely used for evaluating interlinkages between resource systems, trying to identify challenges, synergies, trade-offs and assess holistic solutions \cite{hoffNexusApproachMENA2019,kingRapidAssessmentWater2015}. Nexus thinking is also gaining attention in transboundary resources settings \cite{uneceReconcilingResourceUses2015,kingRapidAssessmentWater2015}. Such is the case of the North Western Sahara Aquifer System (NWSAS). The NWSAS is located in North Africa covering large parts of Algeria, Tunisia and Libya, and it holds invaluable groundwater resources to maintain livelihood in the region \cite{BetterValorizationIrrigation2015}.

Its large extension over an area of more than 1 million km\textsuperscript{2}, makes the NWSAS one of the largest groundwater sources in the world. NWSAS is the main source of water for all socio-economic activities in the region, such as agriculture, industry and domestic use \cite{almullaNWSAS}. Therefore, the growth in the agricultural activities in the past years increased the water abstraction levels significantly. The cropland area reached about 470,000 ha in 2014 \cite{Socioeconomicaspectsirrigation2014} of which 60\% are irrigated by NWSAS water \cite{almullaNWSAS}. The water abstractions, as result, jumped from around 0.6 billion cubic meters (BCM) in 1970 \cite{BetterValorizationIrrigation2015} to 3.2 BCM in 2018 \cite{almullaNWSAS} which is three times higher than the average annual recharge rate of 1 billion m\textsuperscript{3}/year \cite{unepProtectionNorthWest2010}. This overexploitation is pushing for urgent and coordinated actions to safeguard this essential groundwater resource for 4.8 million people \cite{uneceReconcilingResourceUses2020}. Moreover, the NWSAS has suffered improper disposal of non-treated wastewater and irrigation drainage (i.e tailwater), resulting in excessive rise of surface water tables, as the case for Ouargla and El Oued, leading to soil and water salinization \cite{BetterValorizationIrrigation2015}.

The main scientific studies on the NWSAS have been conducted in the form of joint efforts between Algeria, Tunisia, and Libya \cite{abuzeidNorthWesternSahara2015}. Such efforts have identified challenges and risks that the NWSAS has been facing mainly in terms of water scarcity and utilization \cite{BetterValorizationIrrigation2015}. Research outcomes have achieved common databases containing over 9,000 water extraction points, developed hydraulic models to assess impacts of water withdrawals, consultation mechanisms for joint management of water resources, 
% evaluated new water withdrawal potential and the associated risk of future exploitation, 
identified the inefficiency of irrigation, the inadequate valorization of water and the degradation of soil quality in the region \cite{khaterNorthWesternSahara2014,abuzeidNorthWesternSahara2015,BetterValorizationIrrigation2015,Socioeconomicaspectsirrigation2014}. 
% Moreover, technical innovation pilots have been implemented in the NWSAS under four themes \cite{ossAgriculturalDemostrationPilots2014}: 1) solar energy for pumping water, both for irrigation and for drainage evacuation, 2) brackish water valuation through demineralization, 3) rehabilitation of lands degraded due to water stagnation, and 4) irrigation efficiency and agricultural intensification. 

% Although, the joint efforts of the three countries and the Sahara and Sahel Observatory (OSS) have been substantial and key for understanding the complexity and vulnerability of the NWSAS system, the components of their research have been heavily driven by understanding water issues---with a lesser focus on the sectors that drive or are needed to support the supply of water. 
Moreover, \citet{almullaNWSAS} developed the first  water-energy-food (WEF) Nexus analysis in the NWSAS, capturing dynamics between the agricultural, water and energy sectors. 
% The focus was on understanding water requirements for agricultural irrigation, its related energy for pumping demand and exploring solar PV competence to provide such energy under various scenarios. 
The main outcomes of the study are helpful to inform policy making in the three countries and enhance synergies between sectors. However, the aim of the study was not on identifying measures to ease water scarcity, rather than to quantify the WEF Nexus perspective and evaluate options to transition to clean energy sources in agriculture. Most recently, the United Nations Economic Commission for Europe (UNECE) \cite{uneceReconcilingResourceUses2020} conducted a basin-wide water-food-energy-ecosystems Nexus study in the NWSAS, in the form of a participatory assessment. The study yielded high-priority, implementable solutions aligned with the Sustainable Development Agenda. The solutions ranged from governance and international cooperation, to economic and policy instruments, infrastructure and innovation. 

Within the solutions identified in \cite{uneceReconcilingResourceUses2020}, two are of concern of the present study: 1) to set up dedicated policies and related incentives for wastewater reuse in agriculture and urban areas, and 2) to upscale the use of non-conventional water resources through desalination and wastewater and drainage treatment. 
% These solutions are aligned with targets of SDG 6 on ensuring availability and sustainable management of water and sanitation for all. Target 6.3 aims at improving water quality by reducing by half the amount of non-treated wastewater and increasing recycling and safe reuse globally \cite{UNSDGs2019, tortajadaContributionsRecycledWastewater2020}. 
Such solutions are aligned with international recognition on treated wastewater/tailwater reuse as one of the best measures to ease water scarcity \cite{unescoWastewaterUntappedResource2017,GARCIA2015154}, as it could substantially increase availability of clean water resources for all uses not available otherwise. Furthermore, non-treated wastewater that runs freely to the environment, often poses severe environmental and health consequences, polluting groundwater aquifers, rivers, lakes, soils and food, among others \cite{unescoWastewaterUntappedResource2017}.
% Nonetheless, there are several cases of success where treated wastewater reuse has substantially improved agricultural irrigation (aiding food production), incremented water availability for ecosystem services and supported aquifer recharge \cite{hettiarachchiSAFEUSEWASTEWATERa, halalshehPolicyGovernanceFramework2018, mahjoubPublicAcceptanceWastewater2018, zuurbierUseWastewaterManaged2018, hussainSustainableUseManagement2019}.

The potential of reusing treated wastewater has not yet been explored in the basin, neither the synergies and trade-offs it may have with the energy, water and agricultural sectors. In general, wastewater treatment and reuse has been commonly evaluated with water, food and health centred approaches, often overlooking energy requirements and its implications \cite{unescoWastewaterUntappedResource2017,hettiarachchiSAFEUSEWASTEWATERa,cheniniEvaluationThreeDecades2011,qadirNonconventionalWaterResources2007}. On the other hand, wastewater-related nexus approaches, have usually focused on the energy and nutrients recovery potential from wastewater \cite{guestNewPlanningDesign2009,gremillionWastewaterResourceWaterWasteEnergy,unescoWastewaterUntappedResource2017}. To date, a research gap was found into how to determine the potential of wastewater treatment and reuse, assessing possible treatment technologies, the energy demand implications and the effects on the water system. To address that gap, the objective of this study was to develop a novel exploratory WEF Nexus approach to asses the impact that reclaiming, treating and reusing domestic wastewater/agricultural drainage (i.e. in agricultural irrigation), may have in the water, food and energy systems. Tailored Geographic Information Systems (GIS) methods were used with a Levelized Cost of Water (LCOW) methodology \cite{ISI:000209031000003} to capture key spacial characteristics of the Nexus system. Furthermore, the methodology was applied to the NWSAS in the context of sustainable development.

\newpage
\section{Methodology}
% The methods used in this study follow those of the framework presented in \cite{ramirezgomezTechnoeconomicGISbasedModel2018}, with some key additions and improvements. The summary of the methodology and methods is presented in \fref{fig:framework}.
The methods used in this study can be divided into water-related, energy-related and cost-related, with additional supporting processes (\fref{fig:framework}). GIS tools were widely used throughout the analysis, and a general model and case study runner for the NWSAS where developed using Python and hosted in a \href{https://github.com/camiloramirezgo/NWSAS-paper-model}{\textbf{Github repository}}.

\begin{figure*}[!h]
	\centering
	\includegraphics[width=\textwidth]{Framework}
	\caption{Methodology flow diagram---blue boxes indicate water-related methods, red boxes energy-related methods, gray boxes cost and LCOW related methods, orange boxes supporting methods and the yellow cylinders scenario characteristics data.}
	\label{fig:framework}
\end{figure*}

The water-agriculture-energy system evaluated for the NWSAS consisted of the extraction of groundwater resources, desalination of brackish water when needed, water demand for domestic and agricultural irrigation purposes, reclaim of domestic wastewater and agricultural drainage (i.e. tailwater), treatment of reclaimed wastewater, and treated wastewater reuse in agricultural irrigation. A schematic of the system is presented in \fref{fig:system_reuse}.

\begin{figure*}[!ht]
	\centering
	\includegraphics[width=\textwidth]{System_reuse}
	\caption{NWSAS components and resource streamflows - WWR scenarios.}
	\label{fig:system_reuse}
\end{figure*}

All water requirements for population consumption and agricultural irrigation were assumed to be supplied by the groundwater aquifer and that such water is desalinated whenever the TDS levels are above 1,000 mg/l \cite{fao1985water} (according to \textit{section 7} of the \textit{supplementary information}). Afterwards, the water is allocated for population consumption and agricultural irrigation. 
% Finally, wastewater from population and tailwater from agriculture is disposed to the environment without any adequate treatment. 
The recharge rate $R$ for the entire aquifer was taken as 1.1 billion cubic meters of water per year, which for the area of the aquifer, is an equivalent water column of 1.06 mm per year \cite{BetterValorizationIrrigation2015}. Furthermore, no environmental flow was considered.

Six wastewater treatment and reuse scenarios were analysed, evaluating three irrigation water pricing regimes and two levels of population water consumption. Irrigation water pricing regimes were taken from \cite{Socioeconomicaspectsirrigation2014}, were it was found that the irrigation water demand per hectare throughout the NWSAS basin is heavily dependent on the supply water cost. Three different regimes for water users exist: 

\begin{enumerate}
	\item \textbf{Private water users:} private farmers that pay the full price of water without any subsidy. The average level of water demand is around 10,512 m\textsuperscript{3}/ha. The \citet{Socioeconomicaspectsirrigation2014} found, that farmers belonging to this regime, have higher water productivity. 
% 	This means that, they have a better use of the resource, obtaining more product while using less amount of irrigation water per hectare.
	\item \textbf{Subsidized water users:} users that have access to water subsidized to some extent. The average water demand is 15,334 m\textsuperscript{3}/ha.
	\item \textbf{Free water users:} farmers that have free access to water, meaning that the government fully subsidize the price of water and that the resource can be utilized without limitations. The average irrigation water demand is 21,215 m\textsuperscript{3}/ha.
\end{enumerate}

Water demand in the subsidized and free regimes, constitutes a 45.8\% and a 101.8\% increase in irrigation water requirements compared to the private regime respectively. This suggests a strong price elasticity of the irrigation water demand and/or the use of lower efficiency irrigation technologies \cite{Socioeconomicaspectsirrigation2014}.

On the other hand, two levels of population water consumption were analysed based on the work of \citet{Householdwaterconsumption2014}:

\begin{enumerate}
    \item \textbf{Low level:} an average water consumption per capita of 55 m\textsuperscript{3}/year.
    \item \textbf{High level:} an average water consumption per capita of 73 m\textsuperscript{3}/year.
\end{enumerate}

In addition, a sensitivity analysis was performed on the groundwater quality and depth to groundwater levels, in order to asses the impact they pose to the energy-for-water requirements (\tref{tbl:sensitivy}). This parameters were selected, as the water security of the aquifer highly depends on the water table levels and the quality of the resource, directly affecting food security as well.

\begin{table}[!ht]
	\caption{\label{tbl:sensitivy}Sensitivity parameters.}
	\begin{indented}
	\item[]\begin{tabular}{@{}l l l l}
		\br
		Parameter & Low & middle & high\\
		\mr
		Groundwater quality & -10 meters & current level & +10 meters\\
		Depth to groundwater & -50\% & current level & +50\%\\
		\br
	\end{tabular}
	\end{indented}
\end{table}

% \noindent Groundwater quality:
% \begin{itemize}
% 	\item Low: -50\% of the current TDS levels,
% 	\item Neutral: the current TDS levels,
% 	\item High: +50\% of the current TDS levels.
% \end{itemize}

% \textbf{Depth to groundwater:}
% \begin{itemize}
% 	\item Low: -10 meters of the current depth to groundwater levels,
% 	\item Neutral: the current depth to groundwater levels,
% 	\item High: +10 meters of the current depth to groundwater levels.
% \end{itemize}

% \subsection{Geographic Information Systems analysis}

% \begin{table*}[b]
% 	\caption{\label{tbl:datasources}Geographic Information System data sources}
%     {\footnotesize
%     \begin{tabular*}{\textwidth}{@{}P{1.4in} P{1in} l P{1.2in} l l@{}}
%     \br
%     Layer & Coverage & Format & Resolution & Year & Source\\
%     \mr
%     Population & Algeria, Tunisia, Libya & raster (tif) & 100 m grid cell & 2015 & \cite{Worldpop2012}\\\ms
%     Depth to groundwater & Africa & txt table & 5 km grid cell & 2012 & \cite{Quantitativemapsgroundwater2012a}\\\ms
%     Administrative boundaries & Africa & shapefile & Individual country polygons & 2017 & \cite{Humanitarian2017}\\\ms
%     Administrative boundaries & Algeria, Tunisia, Libya & shapefile & Level 1 (provinces) polygons & 2015 & \cite{GADM}\\\ms
%     Transboundary aquifers borders & Global & shapefile & Individual polygons & 2015 & \cite{IGRAC}\\\ms
%     Groundwater quality & NWSAS Basin & data points & 206 data points & 2016 & RA*\\\ms
%     Digital Elevation Data* & Africa & raster (tif) & 1, 3 and 15 arc second & 2014 & \cite{DEM2014}\\\ms
%     Land cover & Africa & raster (tif) & 20 m grid cell & 2016 & \cite{ESA2017}\\\ms
%     Aquifer boundaries & NWSAS basin & shapefile & Individual polygons & - & RA*\\\ms
%     Climate data & Global & raster (tif) & 30 arc second, monthly & 1970-2000 & \cite{WorldClimGlobalClimate}\\
%     \br
%     \end{tabular*}\\
%     ~* Regional Authorities.
%     }
% \end{table*}

The most up-to-date open source data available was collected and processed using the open source software QGIS. A data package comprising the geospatial characteristics of the study area was created, in order to capture key information to make the WEF nexus analysis possible. The relevant layers are identified and described in \textit{section 1} of the \textit{supplementary information}.

% All data layers were converted into matching units, re-projected into the Sud Algerie Degree projection (ESRI: 102592)---This projection was selected as it produces minimal distortions in the analysis area---, re-scaled to the same resolution and, when only individual data points were available, interpolated to extend the data to the entire analysed area (i.e. for the Groundwater quality layer). Furthermore, all layers were merged into a large data frame.

% \subsection{Data calibration}
% The \textit{population} and the \textit{irrigated area} layers were calibrated to match regional statistics. The calibration was performed using the fraction given between the regional statistical data (i.e. total population or total irrigated area), and the sum of all data points of the layer in question. The statistical data was available as per country basis. Thus, the calibration process was performed for the basin areas within each country, using their specific information (see \tref{tbl:regionalstats}).

% \begin{table*}[!h]
% 	\caption{\label{tbl:regionalstats}NWSAS population and irrigated area statistics for year 2015, subdivided per country area inside the basin. Data source: \cite{BetterValorizationIrrigation2015}}
% 	\begin{indented}
% 	\item[]\begin{tabular}{@{}l*{4}{r}}
% 		\br
% 		Parameter & Total & Algeria & Tunisia & Libya\\
% 		\mr
% 		NWSAS Population & 6,376,367 & 4,240,888 & 617,168 & 1,518,311\\
% 		NWSAS Irrigated area (Ha) & 469,529 & 237,485 & 56,547 & 175,497\\
% 		\br
% 	\end{tabular}
% 	\end{indented}
% \end{table*}

% Moreover, to calibrate the irrigated area data, an algorithm was run to ensure that non of the data cells had more than 100\% of its area covered by irrigated land.
 
% \subsection{Population and irrigation water withdrawals}
% The calculation of total water withdrawals $ww_{tot,i}$ was performed according to \eref{eq:waterwithdrawals}. Population water withdrawals were calculated as the product between the population count in each data cell ($Pop_{i}$) and the specific water demand per capita ($wpc_{i}$) for the region. Similarly, withdrawals from irrigated agriculture were calculated as the product between the irrigated area inside each data cell ($IrrArea_{i}$) and the specific water demand per cultivated hectare ($wpha_{i}$) for the region.

% \begin{equation}\label{eq:waterwithdrawals} 
% ww_{tot,i} = Pop_{i}\cdot wpc_{i} +IrrArea_{i}\cdot wpha_{i} 
% \end{equation}

% For the baseline, a level of water withdrawals per capita $wpc$ of 55 cubic meters per year was assumed with a population growth of 1\% per year \cite{Householdwaterconsumption2014}. Moreover, all cropland area within the aquifer was considered to be irrigated by groundwater resources and the water requirements per cultivated hectare $wpha$ to be 13,520 m\textsuperscript{3}/Ha for the Algerian part; 13,266 m\textsuperscript{3}/Ha for Tunisia and 9,134 m\textsuperscript{3}/Ha for Libya, according to data from \cite{Socioeconomicaspectsirrigation2014}. Finally, no growth in irrigated area was considered.

\subsection{Reclaimed wastewater and reused treated wastewater}
The amount of residential recoverable wastewater, was assumed to represent around 70\% of the total residential water consumed \cite{unescoWastewaterUntappedResource2017}. From that share, an additional 10\% was assumed to be lost in the capture, conveyance and treatment processes.
On the other hand, to evaluate the potential of capturing, storing and reusing irrigation tailwater, first the water requirements of the crop were estimated. The FAO-56 Penman-Monteith method \cite{allenFAOIrrigationDrainage1998} was used, calculating meteorological parameters from ``WorldClim" monthly data \cite{WorldClimGlobalClimate} through the Python library ``Pyeto" \cite{pyeto}. For the purpose of this study, date palms were assumed to cover 100\% of the cropland area---as is one of the main crops cultivated in the region. The crop coefficients and irrigation calendar were set according to \citet{almullaNWSAS}. From this process, the yearly crop water needs throughout the entire basin were obtained.

Furthermore, an on-farm storage pond system was evaluated to account for the potential reusable water. For this, a water balance on the on-farm storage was executed following a similar approach to \citet{reinhartSimulatedWaterQuality2019}. First, the maximum irrigation efficiency (i.e. crop water requirements over irrigation water needed) was set to be 80\%, being the remaining 20\% non-recoverable loses. If additional water was available, it was recovered and stored. The storage reservoir surface area, was assumed at 2\% of the cropland area and a standard depth of 3 meters \cite{reinhartSimulatedWaterQuality2019}. Leakage losses in the storage system were set to be 0.9 mm/day, and evaporation loses were calculated using a modified Penman-Monteith method for an open water body \cite{reinhartSimulatedWaterQuality2019}. For this, and albedo, surface height, and surface roughness values of 0.05, 0.002 m, and 0 s/m, respectively were used \cite{princeczarneckijobym.QuantifyingCaptureUse2017}.

\subsection{Groundwater Stress Indicator}
The Groundwater Stress Indicator was used to quantify the current stress of the aquifer \cite{Aqueductglobalmaps2015}. It relates the ratio of water withdrawals due to anthropogenic reasons (e.g. potable water, industrial water, recreational water, irrigation water, etc.), and the total recharge rate of the aquifer (subtracting the environmental stream flow). The groundwater stress indicator is usually calculated as the ratio of groundwater footprint to aquifer area \cite{RegionalGroundwaterStress2013}.

% as per \eref{eq:7}:

% \begin{equation}\label{eq:7} 
% GW = \frac{GF}{A_{A}}
% \end{equation}

% \noindent{Where}:
% \begin{itemize}[label={-}]
% 	\item $GW$: Groundwater Stress indicator. Values below 1 indicate low stress areas, values from 1 to 5 indicate low to middle stressed areas, values from 5 to 10 indicate middle to high stressed areas, values from 10 to 20 indicate high stressed areas and values above 20 indicate extremely high stressed areas.
% 	\item $GF$: groundwater footprint. Identifies the right balance between groundwater use and groundwater replenishment for an area.
% 	\item $A_A$: areal extent of an aquifer throughout a given region.
% \end{itemize}

% The groundwater footprint is calculated as:

% \begin{equation}\label{eq:8} 
% GF = \frac{C}{R-E}\cdot A
% \end{equation}

% \noindent{Where}:
% \begin{itemize}[label={-}]
% 	\item $C$: total area-averaged annual withdrawals of groundwater for anthropogenic use.
% 	\item $R$: total area-averaged annual recharge rate of water for groundwater aquifer, including natural and anthropogenic sources.
% 	\item $E$: total area-averaged annual environmental stream flow used to sustain ecosystem services.
% 	\item $A$: areal extent of a given region where $C$, $R$, and $E$ can be defined.
% \end{itemize}

% \subsection{Groundwater pumping}\label{Sc:pumping}
% The energy needs to pump water from groundwater resources is given by the required lift ($H-h$), the pressure drop due to fluid friction in the piping, and the pressure losses in valves and fittings. Pressure losses due to friction in the piping were found to be rather small compared to the lift requirements. Therefore, and due to lack of specific data on wells and boreholes in the region, the pressure losses due to friction in the piping and in valves and fittings were disregarded. The energy requirements (in watt-h) can then be estimated as \eref{eq:1} \cite{Groundwaterdependentirrigationcosts2017}:

% \begin{equation}\label{eq:1}
% E = \frac{Q\cdot(\rho\cdot g\cdot(H - h))}{\eta}
% \end{equation}

% Where $Q$ stands for the water extractions (m\textsuperscript{3}), $\rho$ for the water density (kg/m\textsuperscript{3}), $g$ for the gravitational acceleration (m/s\textsuperscript{2}), $H$ for the delivered hydraulic head (meters), and $h$ for the head in the well (meters). Moreover, $\eta$ accounts for the pumping efficiency, which was set as 85\% along the entire aquifer.

% \subsection{Reverse Osmosis desalination}\label{Sc:RO}
% Reverse Osmosis (RO) desalination is the most popular desalination technology used worldwide. Its energy intensity falls typically in the range of 0.5 to 2.5 kWh per cubic meter of desalinated brackish water \cite{Energyoptimalgroundwater2013}.

% To estimate the energy required to desalinate one cubic meter of saline water, often detailed information of the RO system is required. When analysing a broad area using a geospatial approach, such information is not available as the characteristics of the system can change from application to application \cite{stillwellPredictingSpecificEnergy2016,aminfardMultilayeredSpatialMethodology2019}. Thus, a simplified approach was used to estimate the RO energy requirements. RO is a pressure-driven process that forces water through a membrane which separates dissolved solutes using preferential diffusion. The output water from the membrane (\textit{permeate, p}) is relatively free of solutes, while the remaining water (\textit{concentrate, c}) exits the pressure vessel with a high concentration of solutes (i.e. high TDS levels). A schematic representation of the process is presented in \fref{fig:ro} \cite{crittenden_mwhs_2012}.

% \begin{figure}[!ht]
% 	\centering
% 	\includegraphics[width=0.4\textwidth]{Reverse_Osmosis}
% 	\caption[Reverse Osmosis schematic separation process]{Reverse osmosis schematic separation process. Based on: \cite{crittenden_mwhs_2012}.}
% 	\label{fig:ro}
% \end{figure} 

% The minimum energy required to push the water through the membrane is given by the amount of diluted solutes in the \textit{feed (f)} water. Such minimum energy can be estimated calculating the osmotic pressure of the \textit{feed} water, as described in \eref{eq:6} \cite{crittenden_mwhs_2012}.

% \begin{equation}\label{eq:6}
% \pi = \phi\cdot C\cdot R\cdot T
% \end{equation}

% \noindent{Where}:
% \begin{itemize}[label={-}]
% 	\item $\pi$: osmotic pressure (bar),
% 	\item $\phi$: osmotic coefficient, close to 1 (-), assumed a 0.95 \cite{crittenden_mwhs_2012},
% 	\item $C$: concentration of all solutes (mol/L),
% 	\item $R$: universal gas constant, 0.083145 (L$\cdot$bar/mol$\cdot$K),
% 	\item $T$: absolute temperature (K), (273 + \degree C), assumed at 25 \degree C for the entire aquifer.
% \end{itemize}

% Thus, the minimum energy demand can be estimated multiplying the osmotic pressure of the \textit{feed} water $\pi$ (in bar) by a conversion factor of 1.0 kWh/m\textsuperscript{3} = 36 bar. In reality, the energy demanded is greater due to factors as friction losses, membrane filtration resistance, among others. However, this approach has been used in cases were no specific data of the RO system is available \cite{KARABELAS201815}.

% The water quality layer used, was obtained from 206 measurements provided by National Authorities of the region. Each point specifies the spacial location and groundwater TDS content. Although the data did not covered the entire basin area, due to lack of any other related information it was used to produced a raster layer. An inverse distance weighted interpolation method, having as distance weighting factor an inverse distance to a power of 2 and a global search radius with maximum number of nearest points of 10 was used (see \fref{fig:TDS}).

% \begin{figure*}[!ht]
% 	\centering
% 	\includegraphics[width=0.88\textwidth, cfbox=black 1pt 0pt]{NWSAS_TDS}
% 	\caption[NWSAS groundwater quality map - Total Dissolved Solids (TDS)]{North Western Sahara Aquifer System - Groundwater quality map, Total Dissolved Solids (TDS) at 1$\times$1 km grid cell resolution.}
% 	\label{fig:TDS}
% \end{figure*}

% \subsection{Energy-for-wastewater}\label{Sc:eww}
% To calculate the energy-for-wastewater requirements an energy intensity factor was used for each evaluated treatment technology following \eref{eq:energy-for-wastewater}.

% \begin{equation}\label{eq:energy-for-wastewater}
% E_{ww} = Q_{ww,yr}\cdot X_t
% \end{equation}

% Where $Q_{ww,yr}$ represents the yearly treated wastewater in m\textsuperscript{3}/yr, and $X_t$ the average energy demand of the specific WWTT $t$, to treat one m\textsuperscript{3} of wastewater (in kWh/m\textsuperscript{3}).

\subsection{Clustering algorithm}\label{Sc:clustering}
A clustering approach was used in order to identify dense areas where a wastewater treatment system could be implemented, minimizing constraints imposed by existent large distances among scatter population or irrigated lands. A hierarchical clustering  algorithm was run using the \textit{Agglomerative Clustering} object from the Python \textit{scikit-learn} package \cite{scikit-learn}.
% Such algorithm relies on a bottom-up approach to define the clusters, in which all data points are first identified as an individual cluster, to be then successively merged together. The linkage criteria used was the \textit{ward} linkage.

\subsection{Wastewater Treatment System characteristics}
% The costs incurred in the implementation of a WWTS, can be divided into three major groups: \begin{enumerate*}[label=\upshape(\arabic*\upshape)]
% 	\item CAPEX, \item OPEX and \item Conveyance or transport system costs.
% \end{enumerate*} The first two, account for capital and operation expenses, and the third one for implementation and operation of the wastewater transportation system. Due to the high complexity of evaluating the costs of a conveyance system, only CAPEX and OPEX costs were considered.

% Statistical methods have shown to be commonly used among cost-modelling for wastewater management \cite{Costmodellingwastewater2011,Assessmentwastewatertreatment2012,Economicfeasibility2012}. Such methods use available cost figures (i.e. historical or estimated cost figures of actual WWTPs as capacity, pollutants treated, etc.) to create cost functions that adjust the cost relevant data to independent variables, describing the behaviour of a dependent variable (i.e. CAPEX and OPEX). The resulting cost function, allows to evaluate the capital and operational costs, often in terms of wastewater flow or served population equivalent \cite{Economicvaluationwastewater2015}. One advantage of this approach from a GIS perspective, is that with the use of non-linear cost functions, the effect of economies of scale can be easily evaluated. Furthermore, as a top-down approach enables for a better understanding of the relationship among variables and wastewater management, it constitutes a scientific approach for new wastewater services planning \cite{Costmodellingwastewater2011}. 

Wastewater pollutant levels, were assumed to be constant throughout the basin, using standard values based on studies from FAO \cite{fao1985water}. The assumed pollutant levels for population wastewater and the required levels for reused treated wastewater, are shown in \textit{section 6} of the \textit{supplementary information}.

% \begin{table*}[!ht]
% 	\caption{\label{tbl:pollutans}Pollutant levels of wastewater and treated wastewater (mg/l).}
% 	\begin{indented}
% 	\item[]\begin{tabular}{@{}l r r}
% 		\br
% 		Pollutant type & Wastewater & Treated wastewater\\
% 		\mr
% 		Suspended solids ($SS$) & 900 & 100\\
% 		Nitrogen ($N$) & 40 & 10\\
% 		Phosphorus ($P$) & 20 & 2\\
% 		Biochemical Oxygen Demand (BOD\textsubscript{5}) ($BOD_5$) & 500 & 50\\
% 		Chemical Oxygen Demand ($COD$)& 500 & 50\\
% 		\br
% 	\end{tabular}
% 	\end{indented}
% \end{table*}

Statistical methods have shown to be commonly used in cost-modelling for wastewater management \cite{Costmodellingwastewater2011,Assessmentwastewatertreatment2012,Economicfeasibility2012}, thus, cost functions were used to estimate CAPEX and OPEX values for the different Wastewater Treatment Technologies (WWTTs). However, technology specific cost functions were not available for the NWSAS basin area, nor statistical data to develop them. Therefore, based on the work of \citet{Assessmentwastewatertreatment2012} cost functions for different WWTT in Spain were used to evaluate the competence of selected technologies in the NWSAS (see \textit{section 6} of the \textit{supplementary information}). Energy intensity characteristics were added for each technology according to \cite{Energypatternanalysis2012,ComparativeAnalysisEnergy2017}.

% % TODO: add description of acronyms
\begin{table*}[!h]
    \caption{\label{tbl:treatmentsystems}Treatment systems analysed. Adapted from \cite{Assessmentwastewatertreatment2012} unless otherwise stated.}
\end{table*}
	~\\[-50pt]{\footnotesize
	\begin{longtable}{@{}P{1.2in} P{0.9in} P{1.9in} P{0.7in} P{0.7in}}
	\br
    Technology & Contaminant removal (\%) & Costs (\euro) & Energy (kWh)$^{\ddagger}$ & Usage\\
    \mr
    \endfirsthead
    \multicolumn{4}{@{}l}{\ldots continued}\\\br
    Technology & Contaminant removal (\%) & Costs (\euro) & Energy (kWh)$^{\ddagger}$ & Usage\\\mr
    \endhead % all the lines above this will be repeated on every page
    \br
    \multicolumn{4}{r@{}}{continued \ldots}\\
    \endfoot
    \endlastfoot
    Pond System (PS) & N: 20 -- 40 \newline P: 60 -- 70 \newline COD: 60 -- 96 \newline SS: 50 -- 90 & CAPEX: $3897.7\cdot x^{-0.407}$ \newline OPEX: $5.543\cdot x + 3127.5$ & $0.19\cdot V$ & Irrigation tailwater\\
    Intermittent Sand Filter (ISF) & N: 65 -- 95 \newline P: 75 -- 99 \newline COD: 75 -- 90 \newline SS: 85 -- 95 & CAPEX: $2115.5\cdot x^{-0.399}$ \newline OPEX: $12.026\cdot x+3518.9$ & $0.2\cdot V$ & Population wastewater\\
    Trickling Filter (TF) & N: 35 -- 50 \newline P: 35 -- 55 \newline COD: 75 -- 90 \newline SS: 50 -- 90 & CAPEX: $12237\cdot x^{-0.87}$ \newline OPEX: $13.504\cdot x+6020$ & $0.3\cdot V$ & Population wastewater\\
    Moving Bed Biofilm Reactor (MBBR) & N: 10 -- 20 \newline P: 30 -- 40 \newline COD: 20 -- 40 \newline SS: 60 -- 80 & CAPEX: $1187\cdot x^{-0.165}$ \newline OPEX: $12.794\cdot x+6031$ & $0.8\cdot V$ & Population wastewater\\
    Rotating Biological Contractors (RBC) & N: 20 -- 80 \newline P: 10 -- 30 \newline COD: 70 -- 93 \newline SS: 75 -- 98 & CAPEX: $6931.4\cdot x^{-0.383}$ \newline OPEX: $313.4\cdot x^{-0.435}$ & $0.8\cdot V$ & Population wastewater\\
    Membrane Bioreactor (MBR) & N: 50 -- 90 \newline P: 20 -- 70 \newline COD: 70 -- 90 \newline SS: 85 -- 99 & CAPEX: $5635.3\cdot x^{-0.352}$\newline $^{\dagger}$OPEX: $2.116\cdot V^{0.713}e^{1.51\cdot SS+0.037\cdot BOD}$ & $0.8\cdot V$ & Population wastewater\\
    Extended Aeration (EA) & N: 50 -- 90 \newline P: 15 -- 70 \newline COD: 70 -- 90 \newline SS: 85 -- 99 & CAPEX: $7946\cdot x^{-0.460}$ \newline $^{\dagger}$OPEX: $169.48\cdot V^{0.454}e^{0.61\cdot SS}$ & $0.6\cdot V$ & Population wastewater\\
    Sequencing Batch Reactor (SBR) & N: 55 -- 90 \newline P: 25 -- 70 \newline COD: 70 -- 90 \newline SS: 85 -- 99 & CAPEX: $8258.9\cdot x^{-0.407}$ \newline OPEX: $309.4\cdot x^{-0.389}$ & $1\cdot V$ & Population wastewater\\
    \br
    \multicolumn{4}{@{}l}{$x$: population equivalent, $x=V\times1500/(400\times365)$, $V$: wastewater flow (m\textsuperscript{3}/yr)} \\
    \multicolumn{4}{@{}l}{N: Nitrogen, P: Phosphorus, COD: Chemical Oxygen Demand, SS: Suspended Solids} \\
    \multicolumn{4}{@{}l}{CAPEX: Capital Expenditure, OPEX: Operating Expenses}\\
    $^{\dagger}$ Taken from \cite{Costmodellingwastewater2011} & & & \\ 
    $^{\ddagger}$ Based on \cite{Energyrequirementswater2012,ComparativeAnalysisEnergy2017} & & & 
    \end{longtable}
	}

\subsection{Levelised Cost of Water}
A proposed LCOW method was used as metric to compare cost-effectiveness among WWTTs. The LCOW assesses the life-cycle cost of delivering one unit (e.g. one cubic meter) of treated wastewater, based on all physical assets and resources required. This concept, is inherited from the LCOE methodology, which applies the same life-cost analysis for one unit of electricity output \cite{prospectscostcompetitive2013}. The LCOW method follows the logic of the LCOE method \cite{prospectscostcompetitive2013,GeospatialLevelizedCost2015}, with pertinent adjustments to the variables used in wastewater treatment systems. Then, the LCOW can be expressed as follows:

\begin{equation}\label{eq:lcow}
LCOW = LCOW_{Inv} + LCOW_{O\&M} + LCOW_{Ext}
\end{equation}

The expression presented in \eref{eq:lcow}, disaggregates the $LCOW$ (\$/m\textsuperscript{3}) value in three components: the cost components due to investment $LCOW_{Inv}$, operation and maintenance $LCOW_{O\&M}$ and externalities $LCOW_{Ext}$. As the CAPEX function comprises all investment components of a wasteater treatment plant, it enables an easy calculation of the $LCOW_{Inv}$ for each WWTT and each region or cluster. \Eref{eq:lcow_inv} describes the process to calculate the $LCOW_{Inv}$.

\begin{equation}\label{eq:lcow_inv}
LCOW_{Inv} = \frac{Inv}{\sum_{t=1}^{T} V_{t}\cdot\gamma^{t}}\cdot\Delta
\end{equation}

Where $Inv$ stands for the CAPEX value, $V_{t}$ for the treated water flow per year $t$ (m\textsuperscript{3}/yr), $\Delta$ for the tax factor and $\gamma^{t}$ represents the discount factor of the project \eref{eq:gamma}. The discount factor can be calculated according to the discount rate $r$, as shown in \eref{eq:gamma}. An appropriate discount rate ($r$) needs to be used to ensure the right amount of return for all sources of long term capital (i.e. equity holders and debt). Often, the proper discount rate used is the WACC, which was assumed at 4\% for this study \cite{prospectscostcompetitive2013}. 

\begin{equation}\label{eq:gamma}
\gamma^{t} = \left(\frac{1}{1+r}\right)^{t}
\end{equation}

The tax factor $\Delta$ includes all effects of the tax related variables, these being the rent tax, depreciation, depreciation period, discount factor and investment tax credit \cite{prospectscostcompetitive2013}. No effects related to the tax factor were considered, thus a tax factor of $\Delta=1$ was used.

The LCOW related to operational costs $LCOW_{O\&m}$ \eref{eq:lcow_om} was computed by using the OPEX values $\omega_{t}$ calculated for each year in each cluster, the treated water flow $V_{t}$ (m\textsuperscript{3}/yr) and the discount factor $\gamma^t$ per year.

\begin{equation}\label{eq:lcow_om}
LCOW_{O\&m} = \frac{\sum_{t=1}^{T} \omega_{t}\cdot V_{t}\cdot\gamma^{t}}{\sum_{t=1}^{T} V_{t}\cdot\gamma^{t}}
\end{equation}

Furthermore, the avoidance of externalities due to discharge of untreated wastewater into ecosystems can be accounted in the LCOW calculation. This parameter tries to account for the effects that pollutants presented in wastewater and tailwater stream flows can have into fresh water bodies, rivers or groundwater aquifers \cite{Assessmentwastewatertreatment2012}. The externalities-related LCOW value $LCOW_{Ext}$ \eref{eq:lcow_ext} can be obtained as follows:

\begin{equation}\label{eq:lcow_ext}
LCOW_{Ext} = \frac{\sum_{p}^{P}\sum_{t=1}^{T} m_{p}\cdot B_p\cdot V_{t}\cdot\gamma^{t}}{\sum_{t=1}^{T} V_{t}\cdot\gamma^{t}}
\end{equation}

Where $m_p$ represents the concentration of pollutant of class $p$ avoided with the treatment of one cubic meter of wastewater (kg/m\textsuperscript{3}), and $B_p$ the environmental benefit of avoiding one kilogram of pollutant $p$ running into the environment (\$/kg). Unfortunately, due to lack of information of environmental effects of wastewater pollutants in the region, this parameter was not considered.

Finally, the project life span was set for 35 years, having as starting year 2015 and ending year 2050.

% \subsection{Least-cost option}
% After the LCOW values were calculated, the WWTT with the lowest $LCOW$ value for each cluster was identified.

\section{Results}
\subsection{Water demand}
The water use in the Baseline scenario was estimated at 5,476 Mm\textsuperscript{3}/yr, with agricultural irrigation accounting for 94\% of the total share (see \fref{fig:water}). In the private agricultural water scenarios, the overall water use was lower than the Baseline scenario (i.e water extrated plus water reused), opposite behaviour to the subsidized and free agricultural water ones. However, due to the reused water share in the subsidized regime (i.e. around 43\%), the overall water extractions were lower than those of the Baseline and close to the ones from the private regime. This suggests that either with the use of more efficient irrigation schemes (i.e. the private regime) or the use of lower efficiency irrigation coupled with water reuse (i.e. subsidized regime), similar results could be achieved.

On the other hand, the free agricultural water regime yielded much larger water extractions, even with water reuse. In fact, the share of reused water accounted for around 34\% of the water usage, lower share than that of the subsidized regime. This was due to the cap set to the on-farm storage area of maximum 2\% share of the cropland area. Therefore, while more recoverable water is available in the free water regime, the storage system cannot hold everything. This could be solved by setting a higher value for the permissible on-farm storage area, however, this would mean that more agricultural area would be used for storage purposes, possibly having trade-offs with target 2.3 of SDG 2 on doubling agricultural productivity and incomes of small-scale food producers.

\begin{figure*}[!t]
	\centering
	\includegraphics[width=\textwidth]{Water}
	\caption{Water usage for all scenarios. At left: reused water after reclaim, treatment and allocation classified by population and irrigation source. At right: overall water extractions classified by population and irrigation use. Percentage bars indicate the share of reused water against the total demand.}
	\label{fig:water}
\end{figure*}

Population water levels do not cause meaningful variations in the overall water use, as agricultural water needs are much more extensive. Nonetheless, with the use of more efficient irrigation schemes, the recoverable water from irrigation drainage decreases, thus populations treated wastewater share on agricultural water usage increases (synergies between SDG 6 and SDG 2).

\subsection{Groundwater Stress}
The Groundwater Stress Indicator was computed aggregating the entire aquifer area (\fref{fig:gws}). A value of 5.01 in the Baseline scenario was obtained, which falls inside the medium-to-high stress category---however, it is known that the stress level in some parts of this area can reach to the extremely high category \cite{herbertGlobalAssessmentCurrent2019}. The private and subsidized agricultural water scenarios, presented a Groundwater Stress Indicator in the category of low-to-medium stress, which suggests a successful reduction on groundwater stress levels. However, the outcomes obtained for the free agricultural water scenarios, are higher at 5.68 and 5.65. Such increase in the stress, is due to the higher water requirements for agricultural irrigation. Synergies and trade-offs within SDG 2 with SDG 6 are clear, specially with targets 6.4 on increasing water-use efficiency and ensuring sustainable withdrawals and supply of freshwater to ease water scarcity and 6.6 on protect and restore water-related ecosystems.

\begin{figure}[!t]
	\centering
	\includegraphics[width=0.7\textwidth]{GWS}
	\caption{Groundwater stress indicator for all scenarios.}
	\label{fig:gws}
\end{figure}
\newpage
\subsection{Least-cost wastewater treatment systems}
The least-cost treatment systems obtained for the low and high population water scenarios, share similarities in the combination of technologies identified (\fref{fig:leastLow} and \fref{fig:leastHigh}). Extended aeration, rotating biological contractors and intermittent sand filter were the least-costly technologies chosen. Nonetheless, differences in the technology shares exist. In the low population water requirements scenarios, wastewater was treated by intermittent sand filters, rotating biological contractors and extended aeration by 4\%, 21\% and 75\% share respectively. Whereas, in the high population water requirements scenario this shares were 2\%, 11\% and 87\%. In general, when lower capacity is required simpler treatment technologies are more cost-effective, as the independent variable of the CAPEX and OPEX cost functions is the available reclaimed wastewater flow.

\begin{figure*}[!ht]
	\centering
	\includegraphics[width=0.88\textwidth, cfbox=black 1pt 0pt]{NWSAS_least-cost_system_cluster}
	\caption{Least-cost wastewater treatment options per cluster---low population water requirements.}
	\label{fig:leastLow}
\end{figure*}

\begin{figure*}[!ht]
	\centering
	\includegraphics[width=0.88\textwidth, cfbox=black 1pt 0pt]{NWSAS_least-cost_system_cluster_high}
	\caption{Least-cost wastewater treatment options per cluster---high population water requirements.}
	\label{fig:leastHigh}
\end{figure*}

The previous is important, as the amount of wastewater available from the agglomerations is key for the calculation of the least costly technology. Therefore, with larger agglomerations, scalable and higher capacity systems could be implemented. The downside however, is that if the costs related to the conveyance system are not evaluated, then the distances among population and/or irrigation points become irrelevant, which is arguably far from reality. Thus, the analysis of more compact clusters, reduces the drawbacks of not calculating the costs related to a wastewater conveyance system.

Overall, the least-cost treatment systems obtained, show an important trade-off, as the best solution is dependent from geospatial factors than can render a specific technology less costly than other in a given region. For example, clusters 0, 2, 37, 30 and 14 use rotating biological contractors in the low population water requirements scenarios, whereas extended aeration in the high population water scenario. Similarly, cluster 22 passes from using intermittent sand filters to biological rotating contractors in the same scenarios.

\subsection{Energy requirements}
Yearly energy requirements for agricultural irrigation are shown in \fref{fig:energyclusters} and additional results can be found in section 11 of the \textit{supplementary information}. The energy demand is considerably larger for the three countries in the northern part of the aquifer (i.e. clusters 7,  8, 25, 30, 31, 14, 13 and 39). This is mainly due to the intense agricultural activity of those regions. However, as shown in \fref{fig:energyclusters}, energy requirements are also affected by depth to groundwater and desalination needs due to TDS contents of the water. The depth to groundwater has a stronger impact on the overall energy requirements, as can be evidenced by clusters 29, 26, 3, 15, 13 and 7, all having lower depth values and being consistently placed to the right side of the diagonal. The opposite is true for clusters 9, 16, 39, 14, 31, 30, 25 and 8, all having higher depth levels, thus, higher energy requirements per cubic meter of water extracted. On the other hand, TDS content seems to have a much weaker effect. This suggests that, increasing water table levels is a important parameter to take into account for energy supply planning. It can significantly affect pumping energy requirements, but as well reduce the water productivity of the region. Nonetheless, reduced quality levels of water (i.e. higher TDS content) could have severe consequences on health and crop productivity, variables that were not accounted for in this study.

\begin{figure*}[!b]
	\centering
	\includegraphics[width=\textwidth]{NexusPlot}
	\caption{Total energy and water requirements per cluster in the Baseline scenario. Clusters are represented by the data points and labeled by its number. TDS contents of water (mg/l) are shown as the size parameter of each data point/cluster. Depth to groundwater (m) is represented by the color ramp.}
	\label{fig:energyclusters}
\end{figure*}

The overall energy related outcomes for all scenarios are shown in \fref{fig:energy}. The energy requirements for groundwater pumping represent the major part of all three activities. Desalination energy, although considerable, is much smaller than pumping energy, this mainly due to the medium TDS levels found throughout the groundwater aquifer---in the 500 - 5000 mg/l in most of the area, see section 7 of the \textit{supplementary information} for more detail. All scenarios apart from the free agricultural water ones, reduced overall energy consumption compared to the Baseline. Such reductions, are achieved by the reuse of treated wastewater in irrigation, as the energy intensity of treatment is substantially lower than the energy intensity for pumping water from the deep aquifer. This shows synergies between SDG 6, SDG 2 and SDG 7, as when more wastewater is collected and treated it can be made available for reuse in agriculture, supporting sustainable food production and efficient irrigation schemes. Moreover, it can reduce energy intensity of the system and promote the use of clean energy sources.

\begin{figure*}[!t]
\includegraphics[width=\textwidth]{Energy}
\caption{Energy requirements for all scenarios with TDS and groundwater depth sensitivity analysis. TDS levels correspond to: $low=0.5\times n$ and $high=1.5\times n$; and groundwater depth levels correspond to: $low=n-10$ and $high=n+10$ meters.}
\label{fig:energy}
\end{figure*}

The sensitivity analysis demonstrates how a change of \rpm10 meters in the depth to groundwater level, has an average variation of circa \rpm5.5\% over the overall pumping energy requirements. Moreover, desalination energy requirements showed variations of $-$53\% to $+$50\%, when changes of \rpm50\% in the TDS levels were considered.

These effects on the energy-for-water requirements, are necessary to be assessed when planning for new water policies and water management strategies. Accordingly, accounting for new energy infrastructure, or potential energy savings could be key for the success of a new policy or solution to the water scarcity problem.

\section{Discussion}
On the discussion  explain well how the findings are relevant for the region, but for other regions in the world as well. This can be covered on a broader perspective of the nexus and touch on the circularity of the resources also mentioning the potential for nutrient recovery and energy generation (cite paper from UNIFLORES). Touch some examples as the Egyptian one, on wastewater resource management. The main story-line here is that by capturing treating and reusing wastewater, a more circular use of the resource is being used, accounting for several uses. The potential connection with SDGs could also be mentioned, as SDG for water and sanitation, SDG for energy, SDG for health and SDG for partnership. The paper on Nexus perspectives towards the SDGs can be usd as reference and use the 5 node methodology proposed there to analyse the implications on a systems perspective. The land use and water recovery systems can be analysed, pointing at the interlinkage between land use and water storage. Also analyse the implications on water and food security, both for the region an in a wider context. Then talk about the implications on energy consumption and the connection with climate change due to the high fossil fuel based power sector of the region. Finally finish by acknowledging the potential of treating and reusing wastewater, but the need of having strong water management and pricing mechanisms coupled with technological improvements on agricultural practices as well as the perception of wastewater as a resource, so water stress can be really alleviated in the region and more generally in the world. It is important to discuss as well the importance of cooperation in the region for the proper management of resources (included wastewater).

\section{Conclusion}

\newpage
\newcommand{\newblock}{}
\bibliography{References}
\bibliographystyle{unsrtnat}

\end{document}

